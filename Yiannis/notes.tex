\newcommand{\sig}{\sigma}
\newcommand{\eps}{\epsilon}
\newcommand{\del}{\delta}
\newcommand{\ah}{\alpha}
\newcommand{\lam}{\lambda}
\newcommand{\gam}{\gamma}
\newcommand{\kap}{\kappa}
\newcommand{\rarr}{\rightarrow}
\newcommand{\larr}{\leftarrow}
\newcommand{\ol}{\overline}
\newcommand{\mbb}{\mathbb}
\newcommand{\contra}{\Rightarrow\Leftarrow}
% for cross product
\newcommand{\lc}{\langle} %<
\newcommand{\rc}{\rangle} %>

%other shortcuts
\newcommand{\ben}{\begin{enumerate}}
\newcommand{\een}{\end{enumerate}}
\newcommand{\beq}{\begin{quote}}
\newcommand{\enq}{\end{quote}}
\newcommand{\hsone}{\hspace*{1cm}}
\newcommand{\hstwo}{\hspace*{2cm}}

\newcommand{\noi}{\noindent}
\parskip 5pt
\parindent 0pt

\documentclass[a4paper]{article}
\usepackage{amsmath,amssymb,algorithmic}
\begin{document}
\title{The Confluence of Computational Vision and Computational Linguistics  CS828U Spring '12}
\author{Angjoo Kanazawa}
\maketitle

\section{January 30th Class 1}

\paragraph{Perception vs Language}

Perception is to understand the world through signals. Understanding
is recursive, because you only understand in respect to something
else. When does it stop? 

Perception is a set of algorithms that are operating on a system. What
do they do? The experts can't agree.

David Marr said the goal of perception is to assign labels to objects
in the data. Lead to the field of Computer Vision. This hasn't gone anywhere because there's no interaction with the world.

Difference bewteen perception and understanding: human can perceive
and relate the situation with a set of symbols in his head (the goat,
monkey, ed, tom and the mars bar).

We may never know how animals perceive the world, the contexted needed
to relate with the world. 

Use language to relate things in the world, as well as the nature of
those representations (of signals)

\paragraph{The Cognitive Dialogue}

Query using a lexicon to a vision system. The vision system answers
the query i.e. ``is there a car?'', then the lexicon searches for more
hypotheses. 
Can apply this dialog to any problem, so this is a ( general model for
intelligence
Perception cannot start bottomw up..
log-polar space normalizes the space so that it's scale invariant

\section{February 6th 2012 - Lecture 2}
\label{sec:lecture-2}
\textbf{Concepts}: An apple is a concept, we don't have a memory of a
specific apple. It's like an idea.

Three approaches:
\begin{enumerate}
\item Bayesian: Build a memory of concepts by looking at many.
\item Attribute: Every object have a vector of properties.
\item Theory: A concept is such a complicated thing, it's not just one
  function or datastructure, it's more like a scientific theory. A
  formal system defined on a set of objects and the relationship
  between such objects.
\end{enumerate}

\subsection{Douglas Stay: Visual Filter}
\label{sec:douglas-stay:-visual}
in PASCAL, people start with superpixel segmentation. Problems with
this is that where do you divide the pieces? What happens with
occlusion? How to figure out texture vs object?

A different approach, is by using a filter:
\begin{enumerate}
\item Take a feature (multiscale patch compressed with PCA in $\mathbf{R}^k$)
\item Train a NN classifier
\item Run the classifier on the images, return to step 1 but with
  results from step 2 (feature now in $\mathbf{R}^{2k}$), with a new NN
\end{enumerate}

This iteration is the key because it actively improves/motivates the
classification.

Finding objects people are holding in image streams: Using background
substraction (motion cue) + human filter = figure out objects.

Seems like we recognize objects instantly without reasoning about why
it maybe a ``cup''.

Can we combine filteres to make new ones? By using intersets and
unions. Do we have a limited number of concepts and combine them to
make sentences. The ability to generalize from categories.

A draw back: extremely supervised.

\subsection{Intelligence}
Action is fundamental. A very important part of intelligence is the
ability to understand others, understanding human action.

What's involved in human action? 
\begin{itemize}
\item Movement of body parts
\item Object to act upon
\item Tools
\item Goal
\end{itemize}

\subsection{Chomsky}
\label{sec:chomsky}
Minimalist Program: $X, X', X''$, $X$ a word, $X'$ is a word when we
know something, like a verb, $X''$ when you know a lot about the
word. 

This grammar works for vision as well, instead of words, we'll use
action. Human action has syntax.










\end{document}
 
