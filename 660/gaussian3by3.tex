\newcommand{\sig}{\sigma}
\newcommand{\eps}{\epsilon}
\newcommand{\del}{\delta}
\newcommand{\ah}{\alpha}
\newcommand{\lam}{\lambda}
\newcommand{\gam}{\gamma}
\newcommand{\rarr}{\rightarrow}
\newcommand{\larr}{\leftarrow}
\newcommand{\ol}{\overline}
\newcommand{\mbb}{\mathbb}
\newcommand{\contra}{\Rightarrow\Leftarrow}

%other shortcuts
\newcommand{\ben}{\begin{enumerate}}
\newcommand{\een}{\end{enumerate}}
\newcommand{\beq}{\begin{quote}}
\newcommand{\enq}{\end{quote}}
\newcommand{\hsone}{\hspace*{1cm}}
\newcommand{\hstwo}{\hspace*{2cm}}

\newcommand{\noi}{\noindent}
\parskip 5pt
\parindent 0pt

\documentclass[a4paper,11pt]{article}
\usepackage{amsmath,amssymb}
\begin{document}
\title{Gaussian Elimination}
\author{Angjoo Kanazawa}
\maketitle

\section{Gaussian Elimination and LU Factorization}

\textbf{Problem Statement:} Solve $Ax=b$, where $A$ is a $N$ by $M$
non-singular matrix. 

\noi
\textbf{Goal:} Compute $L$, a unit lower-triangular matrix and $U$ an
upper-triangular matrix s.t. with a $P$ permutation matrix, $A$ is
defined by $A=LU$. Then, we can solve $Ax=b$ by 
\begin{align*}
  Ax&=b\\
  LUx&=b\\
  L\hat b&=b \text{ where } \hat b=L^{-1}b\\
&\Rightarrow Ux=\hat b
\end{align*}
Where since $L$ is a lower triangular matrix, we can solve for $\hat
b$ using forward-substitution in $\mathcal{O}(n^2)$, and similarly,
$U$ is an upper triangular matrix so using backward-substitution we
can solve for $x$ in $\mathcal{O}(n^2)$

\noi
Doing the $LU$ factorization costs $\mathcal{O}(\frac{2}{3}n^3) =
\mathcal{O}(n^3)$.
\section{How to do LU Factorization}
\label{sec:hwo}
\begin{enumerate}
\item Add multiple of first row of $A$ and first entry of $b$ to the
  second to last rows to produce zeros on the first
  column. i.e. Multpily $A$ with $L_1$ to get $A^{(1)}$ ($A^{(i)}$ is
  our $A$ after $i$th many steps).
\item Repeat, then we will get our upper triangular matrix 
  $U=A^{(n-1)}=L_{n-1}\cdots L_1A$, and our $L=L_{1}^{-1}\cdots
  L_{n-1}^{-1}$.

(Why?) because $A=L_1^{-1}L_1A=L_1^{-1}A^{(1)}$ where
$A^{(1)}=L_1A=U_1$ we get: 
  \begin{align*}
A&= (L_{1}^{-1}\cdots L_{n-1}^{-1})(L_{n-1}\cdots L_1A) \\
&=(L_{1}^{-1}\cdots L_{n-1}^{-1})A^{(n-1)}= LU    
  \end{align*}
\end{enumerate}

\section{Step-by-Step example with 3 by 3 case}
\label{sec:example}
Let's walk through the entire thing with a 3 by 3 $A$:
$$Ax=b \Rightarrow
\begin{pmatrix}
  a_{11}&  a_{12}&  a_{13}\\
  a_{21}&  a_{22}&  a_{23}\\ 
 a_{31}&  a_{32}&  a_{33}\\
\end{pmatrix}
\begin{pmatrix}
  x_1\\ x_2 \\ x_3
\end{pmatrix}
=
\begin{pmatrix}
  b_1\\ b_2 \\ b_3
\end{pmatrix}
$$
Where $x$ is unknown. % In an augmented matrix this is:
% $$\begin{pmatrix}
%   a_{11}&  a_{12}&  a_{13}& b_1\\
%   a_{21}&  a_{22}&  a_{23}&b_2\\ 
%  a_{31}&  a_{32}&  a_{33} & b_3\\
% \end{pmatrix}
% $$

\begin{enumerate}
\item
Formally, our $L_1$ is $(\begin{smallmatrix}
  1 &  0&0\\
  -a_{21}/a_{11}& 1 & 0 \\
  -a_{31}/a_{11}& 0 & 1 \\
  \end{smallmatrix})$, so 
  \begin{align*}
    L_1A &=\begin{pmatrix}
  1 &  0&0\\
  -a_{21}/a_{11}& 1 & 0 \\
  -a_{31}/a_{11}& 0 & 1 \\
  \end{pmatrix}\begin{pmatrix}
  a_{11}&  a_{12}&  a_{13}\\
  a_{21}&  a_{22}&  a_{23}\\ 
 a_{31}&  a_{32}&  a_{33}\\
\end{pmatrix}\\
&=\begin{pmatrix}
  a_{11}&  a_{12}&  a_{13}\\
\frac{-a_{21}}{a_{11}}a_{11}+a_{21}& \frac{-a_{21}}{a_{11}}a_{12}+a_{22}&   \frac{-a_{21}}{a_{11}}a_{13}+a_{23}\\ 
 \frac{-a_{31}}{a_{11}}a_{11}+a_{31}&  \frac{-a_{31}}{a_{11}}a_{12}+a_{32}&  \frac{-a_{31}}{a_{11}}a_{13}+ a_{33}\\
\end{pmatrix}\\
&=\begin{pmatrix}
  a_{11}&  a_{12}&  a_{13}\\
0& a_{22}^{(1)}&   a_{23}^{(1)}\\ 
0&  a_{32}^{(1)}&  a_{33}^{(1)}\\
\end{pmatrix}
  \end{align*}

\item $ L_2L_1A=L_2U^{(1)}=$  \begin{align*}
    &\begin{pmatrix}
  1 &  0&0\\
  0 & 1 & 0 \\
0&  -a_{32}^{(1)}/a_{22}^{(1)}& 1 \\
  \end{pmatrix}\begin{pmatrix}
  a_{11}&  a_{12}&  a_{13}\\
0& a_{22}^{(1)}&   a_{23}^{(1)}\\ 
0&  a_{32}^{(1)}&  a_{33}^{(1)}\\
\end{pmatrix}\\
&=\begin{pmatrix}
  a_{11}&  a_{12}&  a_{13}\\
0& a_{22}^{(1)}& a_{23}^{(1)}\\ 
0 & \frac{-a_{32}^{(1)}}{a_{22}^{(1)}}a_{22}^{(1)}+a_{32}^{(1)}& \frac{-a_{32}^{(1)}}{a_{22}^{(1)}}a_{23}^{(1)}+a_{33}^{(1)}
\end{pmatrix}\\
&=\begin{pmatrix}
  a_{11}&  a_{12}&  a_{13}\\
0& a_{22}^{(1)}& a_{23}^{(1)}\\ 
0&  0&  a_{33}^{(2)}\\
\end{pmatrix} = U\\
  \end{align*}
\end{enumerate}

For all of $L_i$'s we use, $L_i^{-1} = -L_i$ (check it). So our 
$L=L_1^{-1}L_2^{-1} =$
$$\begin{pmatrix}
  1&0&0\\
  a_{21}/a_{11}& 1 & 0\\
a_{31}/a_{11}& a_{32}^{(1)}/a_{22}^{(1)}& 1 \\
\end{pmatrix}
$$ Or just all of the elements of $-L_i$s.

(Remember that inverse/multiplication of a lower triangular matrix is also a lower
triangular matrix)

\noi
long... So now, we have:
\begin{align*}
A &= LU  \\
&=\begin{pmatrix}
  1&0&0\\
  a_{21}/a_{11}& 1 & 0\\
a_{31}/a_{11}& a_{32}^{(1)}/a_{22}^{(1)}& 1 \\
\end{pmatrix}\begin{pmatrix}
  a_{11}&  a_{12}&  a_{13}\\
0& a_{22}^{(1)}& a_{23}^{(1)}\\ 
0&  0&  a_{33}^{(2)}\\
\end{pmatrix}
\end{align*}

\section{Pivoting}
If $a_{11} = 1$, none of that will work. In such a case we need to
permute the first row of $A$ with another row of $A$ with
$max_j|a_{j1}|$.
This is the same as multiplying $Ax=b$ with a \emph{permutation
  matrix}, $P_1$, where all the diagnoal values are 1 but $a_{11},
a_{jj}$ and rest all 0 but $a_{1j}$ and $a_{j1}$

$$P_1 = 
\begin{pmatrix}
  \mathbf{0} &  \cdots &  \cdots & \cdots &\mathbf{1}=a_{1j} & \cdots & 0\\
0 & 1 & 0& \cdots & \cdots & \cdots &  0\\
0 & 0 & 1 &\cdots & \cdots & \cdots &  0\\
\vdots\\
\mathbf{1}_{(a_{j1})} & 0 & 0 & \mathbf{0}_{(a_{jj})} & \cdots & \cdots &  0\\
\vdots\\
0 &  \cdots  & \cdots &\cdots &\cdots &   \cdots &    1\\
\end{pmatrix}
$$

So in the end, we do the same as above and compute $L$, $U$, adn
$P=P_1P_2\cdots P_{k}$ where $k\le n-1$ and get $PA=LU$. Note that there is no
mutliplication/divsion in doing $PA$.
\end{document}
