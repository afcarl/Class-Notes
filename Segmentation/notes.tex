\newcommand{\sig}{\sigma}
\newcommand{\eps}{\epsilon}
\newcommand{\del}{\delta}
\newcommand{\ah}{\alpha}
\newcommand{\lam}{\lambda}
\newcommand{\gam}{\gamma}
\newcommand{\kap}{\kappa}
\newcommand{\rarr}{\rightarrow}
\newcommand{\larr}{\leftarrow}
\newcommand{\ol}{\overline}
\newcommand{\mbb}{\mathbb}
\newcommand{\contra}{\Rightarrow\Leftarrow}
% for cross product
\newcommand{\lc}{\langle} %<
\newcommand{\rc}{\rangle} %>

%other shortcuts
\newcommand{\ben}{\begin{enumerate}}
\newcommand{\een}{\end{enumerate}}
\newcommand{\beq}{\begin{quote}}
\newcommand{\enq}{\end{quote}}
\newcommand{\hsone}{\hspace*{1cm}}
\newcommand{\hstwo}{\hspace*{2cm}}

\newcommand{\noi}{\noindent}
\parskip 5pt
\parindent 0pt

\documentclass[a4paper]{article}
\usepackage{amsmath,amssymb,algorithmic}
\begin{document}
\title{Image Segmentation CS828 Spring '12}
\author{Angjoo Kanazawa}
\maketitle
\section{January 25th First Class}
\textbf{Definition}: Segmentation (for this class):
\begin{itemize}
\item About low level vision in general
\item Requires a lot of knowledge about th eworld, high level understanding,
quite challenging.
\item  So we're going to focus on simpler segmentation
that doesn't require that much knowledge about the world: Uniform
surfaces, smooth shape. Still there will be varietion in intensity.
\item Want to find uniform region in things (texture, color, motion, smoothness), not necessarily world property. Removed from true segmentation of objects but still useful.
\item Image is an 2D geometric structure. Segmentation is clustering
  that takes advantage of this structure. Based on the assumption that
  near-by pixels have the same intensity. 
\item 
\end{itemize}

We're going to look at
\begin{enumerate}
\item Diffusion
\item Anisotropic diffusion
\item Graph based algorithms: message passing, thinking of an image as
  a graph, every pixel is a node in a graph, edges to neighbors
  $\rarr$ Markov Random FIeld. Gives us a probablistic way to express
  the state of a node in relation to its neighbors. Usually NP-hard, but graph-cut and
  belief propagation algorithms still work. The biggest issue is when
  the number of labels is big.
\item Conditional Random Fields, a general version of MRF
\item Normalized Cut: form a graph
\item Wavelets
\end{enumerate}

\textbf{Math}
\begin{itemize}
\item Fourier transforms
\item Convolution
\item Diffusion
\item Wavelets
\item Level sets
\item Riemannian Geometry
\end{itemize}

\textbf{Current Research}
\begin{itemize}
\item Bilateral filtering (by Morel)
\item Texture Segmentation
\item Cosegmentation
\item Affinity propagation
\item 
\end{itemize}

\textbf{Workload}
\begin{enumerate}
\item Reports (6 out of 8 papers):Be critical when reading papers, even if the paper is good, what is
the really important. Learn to recognize, have a taste. (10\%)
\item Presentations: 3 presentations per day, 15 min per paper 10 min
  each to discuss paper (15\%)
\item a take home midterm, Final all on lecture material (50\%)
\item Problem set/Project (25\%)
\end{enumerate}

\end{document}
