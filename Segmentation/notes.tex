\newcommand{\sig}{\sigma}
\newcommand{\eps}{\epsilon}
\newcommand{\del}{\delta}
\newcommand{\ah}{\alpha}
\newcommand{\lam}{\lambda}
\newcommand{\gam}{\gamma}
\newcommand{\kap}{\kappa}
\newcommand{\rarr}{\rightarrow}
\newcommand{\larr}{\leftarrow}
\newcommand{\ol}{\overline}
\newcommand{\mbb}{\mathbb}
\newcommand{\contra}{\Rightarrow\Leftarrow}
% for cross product
\newcommand{\lc}{\langle} %<
\newcommand{\rc}{\rangle} %>

%other shortcuts
\newcommand{\ben}{\begin{enumerate}}
\newcommand{\een}{\end{enumerate}}
\newcommand{\beq}{\begin{quote}}
\newcommand{\enq}{\end{quote}}
\newcommand{\hsone}{\hspace*{1cm}}
\newcommand{\hstwo}{\hspace*{2cm}}

\newcommand{\noi}{\noindent}
\parskip 5pt
\parindent 0pt

\documentclass[a4paper]{article}
\usepackage{amsmath,amssymb,algorithmic}
\begin{document}
\title{Image Segmentation CS828 Spring '12}
\author{Angjoo Kanazawa}
\maketitle
\section{January 25th First Class}
\textbf{Definition}: Segmentation (for this class):
\begin{itemize}
\item About low level vision in general
\item Requires a lot of knowledge about th eworld, high level understanding,
quite challenging.
\item  So we're going to focus on simpler segmentation
that doesn't require that much knowledge about the world: Uniform
surfaces, smooth shape. Still there will be varietion in intensity.
\item Want to find uniform region in things (texture, color, motion, smoothness), not necessarily world property. Removed from true segmentation of objects but still useful.
\item Image is an 2D geometric structure. Segmentation is clustering
  that takes advantage of this structure. Based on the assumption that
  near-by pixels have the same intensity. 
\item 
\end{itemize}

We're going to look at
\begin{enumerate}
\item Diffusion
\item Anisotropic diffusion
\item Graph based algorithms: message passing, thinking of an image as
  a graph, every pixel is a node in a graph, edges to neighbors
  $\rarr$ Markov Random FIeld. Gives us a probablistic way to express
  the state of a node in relation to its neighbors. Usually NP-hard, but graph-cut and
  belief propagation algorithms still work. The biggest issue is when
  the number of labels is big.
\item Conditional Random Fields, a general version of MRF
\item Normalized Cut: form a graph
\item Wavelets
\end{enumerate}

\textbf{Math}
\begin{tabular}{l c r}
Fourier transforms &  Convolution & Diffusion\\
Wavelets & Level sets & Riemannian Geometry  \\
\end{tabular}

\textbf{Current Research}
\begin{tabular}{l r}
Bilateral filtering (by Morel) & Texture Segmentation\\
Cosegmentation & Affinity propagation\\
\end{tabular}

\textbf{Workload}
\begin{enumerate}
\item Reports (6 out of 8 papers):Be critical when reading papers, even if the paper is good, what is
the really important. Learn to recognize, have a taste. (10\%)
\item Presentations: 3 presentations per day, 15 min per paper 10 min
  each to discuss paper (15\%)
\item a take home midterm, Final all on lecture material (50\%)
\item Problem set/Project (25\%)
\end{enumerate}

\section{January 30th Class 2}

\subsection{Perceptual Grouping}

\begin{itemize}
\item Putting pieces to preceive as a whole.
\item Depends on the prior knowledge/statistics about the world.
\end{itemize}

\paragraph{History}
  \begin{itemize}
  \item Behaviorists dominated in early 20th century, wanted to make
    psychology scientific, focused on quantifyiable things.
  \item Rejected anything introspective or mind building internal representations.
  \item AI, computers, chomsky killed behaviorists.
  \item Gestalt movement claimed visual system perceived world as a
    objects and surfaces, as a whole and not as raw atomic stimulus/intensities.
  \end{itemize}
\paragraph{Classical principles/cues}
  \begin{itemize}
  \item Knowing the role of edges is critical to how we perceive an image
  \item Similarity, Good continuation, Common Form, Connectivity, Symmetry (seems
    to jump out), Convexity, Closure, Common Fate, Paraallelism, Collinearity
  \item convexity beats symmetry? Connectivity also beats symmetry?
  \end{itemize}

  \paragraph{Theories}
  \begin{itemize}
  \item We perceive shapes that are ``good form'': smooth curves,,
    pretty abstract
  \item Bayesian: organizaton that's most likely to be true. Not
    computationally friendly. Rather than checking all possible
    options, maybe we look for a certain small set of
    possiblities. Still doesn't explain everything
  \item 
  \end{itemize}
  

\end{document}
 
